\documentclass{article}
\usepackage{amsmath}
\usepackage{amssymb}
\usepackage[margin=1in]{geometry}
\usepackage{fancyhdr}
\usepackage[most]{tcolorbox} % Load the tcolorbox package with additional libraries

\newtcolorbox{bb}[1][]{
  colback=green!5!white,
  colframe=teal!50!black,
  title=#1,
  boxrule=0.8mm,      
  top=6mm,             
  bottom=6mm,          
  left=6mm,            
  right=6mm,           
  before=\vspace{5mm}, 
  after=\vspace{5mm}   
}

\pagestyle{fancy}
\fancyhf{}
\chead{\textbf{Math 164 - Assignment \#6, Fall 2024}}
\renewcommand{\headrulewidth}{0.4pt}

\begin{document}


\begin{enumerate}
\begin{bb}
    \item  Suppose that we perform an experiment to calculate the gravitational constant $g$ as follows. We drop a ball from a certain height and measure its distance from the original point at certain time instants. The results of the experiment are shown in the following table.
    
    \[
    \begin{array}{|c|c|c|c|}
    \hline
    \text{Time (s)} & 1.00 & 2.00 & 3.00 \\
    \hline
    \text{Distance (m)} & 5.00 & 19.5 & 44.0 \\
    \hline
    \end{array}
    \]
    
    The equation relating the distance $s$ and the time $t$ at which $s$ is measured is given by
    \[
    s = \frac{1}{2} g t^2.
    \]
    
    \begin{enumerate}
        \item[(a)] Find a least-squares estimate of $g$ using the experimental results from the table above.
        \item[(b)] Suppose that we take an additional measurement at time $4.00$ and obtain a distance of $78.5$. Use the RLS algorithm to calculate an updated least-squares estimate of $g$.
    \end{enumerate}
\end{bb}

\textbf{Solution: }\\
Recall that Theorem 12.1 states that the unique vector $x^*$ that minimizes $||Ax-b||^2$ is given by the solution to the equation
\[ A^TAx = A^Tb \implies x^* = (A^TA)^{-1} A^T b\]
To solve this problem, we must first write it in the form of a least squares problem.
\begin{align*}
    s &= \frac{t^2}{2}g \\
    A &= \begin{bmatrix}
        1/2 \\ 2 \\ 9/2
    \end{bmatrix}\\
    b &= \begin{bmatrix}
        5 \\ 19.5 \\ 44
    \end{bmatrix} \\
\end{align*}
So the least squares estimate of g is now given by 
\[
    g^* = (\begin{bmatrix}
        1/2 & 2 & 9/2
    \end{bmatrix}\\\begin{bmatrix}
        1/2 \\ 2 \\ 9/2
    \end{bmatrix}\\)^{-1}\begin{bmatrix}
        1/2 & 2 & 9/2
    \end{bmatrix}\\\begin{bmatrix}
        5 \\ 19.5 \\ 44
    \end{bmatrix} = 9.776
    \]


\bigbreak
\begin{bb}
\item
Let $[x_1, y_1]^\top, \dots, [x_p, y_p]^\top$, $p \geq 2$, be points in $\mathbb{R}^2$. We wish to find the straight line of best fit through these points (``best'' in the sense that the total squared error is minimized); that is, we wish to find $a^*, b^* \in \mathbb{R}$ to minimize
\[
f(a, b) = \sum_{i=1}^p (a x_i + b - y_i)^2.
\]

Assume that the $x_i$, $i = 1, \dots, p$, are not all equal. Show that there exist unique parameters $a^*$ and $b^*$ for the line of best fit, and find the parameters in terms of the following quantities:

\[
\overline{X} = \frac{1}{p} \sum_{i=1}^p x_i, \quad
\overline{Y} = \frac{1}{p} \sum_{i=1}^p y_i, \quad
\overline{X^2} = \frac{1}{p} \sum_{i=1}^p x_i^2,
\]

\[
\overline{Y^2} = \frac{1}{p} \sum_{i=1}^p y_i^2, \quad
\overline{XY} = \frac{1}{p} \sum_{i=1}^p x_i y_i.
\]
\end{bb}

\textbf{Solution: } 

We can rewrite the problem as:
\begin{align*}
    f(a,b) &= \sum_{i=1}^{p}(ax_i + b - y_i)^2 \\
    &= ||\begin{bmatrix}
        x_1 & 1 \\
        x_2 & 1 \\
        \vdots & \vdots \\
        x_p & 1
        \end{bmatrix}
        \begin{bmatrix}
            a \\ b
        \end{bmatrix} - \begin{bmatrix}
            y_1 \\ y_2 \\ \vdots \\ y_p
        \end{bmatrix}||^2 = ||Ax-b||^2
\end{align*}
By Theorem 12.1, the matrix A must be invertible. Since $x_i$'s are distinct, the columns of the matrix are linearly independent, making it full rank and thus invertible. 
    \\Therefore, there exists a unique pair of parameters $a^*, b^*$ that minimizes the total squared error.
    \\Now, let's apply theorem 12.1 to find the mimimizer $x^*$

\begin{align*}
x^* &= (A^TA)^{-1} A^T b \\
 A &= \begin{bmatrix}
    x_1 & 1 \\
    x_2 & 1 \\
    \vdots & \vdots \\
    x_p & 1
    \end{bmatrix} \\ b&=\begin{bmatrix}
        y_1 \\ y_2 \\ \vdots \\ y_p
    \end{bmatrix} \\
    x^* &= (\begin{bmatrix}
        \overline{X^2} & \overline{X} \\ \overline{X} & 1 
    \end{bmatrix})^{-1} \begin{bmatrix}
        \overline{XY}
        \\ \overline{Y}
    \end{bmatrix} \\
    \begin{bmatrix}
        a^*\\ b^*
    \end{bmatrix} &= \frac{1}{\overline{X^2} - \overline{X}^2} \begin{bmatrix}
        \overline{XY} - (\overline{X}) (\overline{Y}) \\
        \overline{X^2}\overline{Y} - (\overline{X}) \overline{XY}
    \end{bmatrix}
\end{align*}

\bigbreak

\begin{bb}
\item We are given a point $[x_0, y_0]^\top \in \mathbb{R}^2$. Consider the straight line on the $\mathbb{R}^2$ plane given by the equation $y = mx$. Using a least-squares formulation, find the point on the straight line that is closest to the given point $[x_0, y_0]$, where the measure of closeness is in terms of the Euclidean norm on $\mathbb{R}^2$. 

\textbf{Hint:} The given line can be expressed as the range of the matrix $A = [1, m]^\top$.
\end{bb}

\textbf{Solution: }\\
By the hint, we know that every point on the line can be expressed as $$v = cA = c\begin{bmatrix} 1 \\ m \end{bmatrix}, c\in \mathbb{R} $$
We can now transform the given problem into the form $||Ax-b||^2$, and apply Theorem 12.1

The measure of closeness is in terms of the Euclidean norm on $\mathbb{R}^2$, which is given by $$||x - x_0||^2 = ||\begin{bmatrix} 1 \\ m \end{bmatrix}c - \begin{bmatrix} x_0 \\ y_0 \end{bmatrix}||^2$$
\begin{align*}
    c^* &= (A^TA)^{-1} A^T b \\
    A &= \begin{bmatrix}
        1\\ m
    \end{bmatrix} \\ b&=\begin{bmatrix}
        x_0 \\ y_0
    \end{bmatrix} \\
    c^* &= \frac{x_0+y_0m}{1+m^2} \\
    v^* &= c^*A = \begin{bmatrix}
        \frac{x_0+y_0m}{1+m^2} \\ \frac{m(x_0+y_0m)}{1+m^2}
    \end{bmatrix}
\end{align*}


\bigbreak

\begin{bb}
    \item 
    Let $A \in \mathbb{R}^{m \times n}$, $b \in \mathbb{R}^m$, $m \geq n$, and $\text{rank} A = n$. Consider the constrained optimization problem
    \[
    \text{minimize } \frac{1}{2} x^\top x - x^\top b \quad \text{subject to } x \in R(A),
    \]
    where $R(A)$ denotes the range of $A$. Derive an expression for the global minimizer of this problem in terms of $A$ and $b$.    
\end{bb}
\textbf{Solution:}\\
We can rewrite the problem as a least squares problem, and then apply Theorem 12.1 \\
Observe that $$ \frac{1}{2} x^\top x - x^\top b = \frac{1}{2}||x-b||^2 $$
And if $x \in R(A)$, then $\exists y \in \mathbb{R}^n$ such that $x = Ay$ \\
The problem is now to find the unique vector $x^*$ that minimizes:  $\frac{1}{2}||Ay-b||^2$

\begin{align*}
    y^* &= (A^TA)^{-1} A^T b \\
    x^* &= Ay^* = A \left( A^TA \right)^{-1} A^T b    
\end{align*}

Note: another interesting way to solve this problem is to express the objective function in terms of $y$, and minimize it by setting the gradient to $0$ and solving for $y^*$.

\bigbreak

\begin{bb}
    \item Let $A \in \mathbb{R}^{m \times n}$, $b \in \mathbb{R}^m$, $m \leq n$, $\text{rank} A = m$, and $x_0 \in \mathbb{R}^n$. Consider the problem
    \[
    \text{minimize } \|x - x_0\| \quad \text{subject to } Ax = b.
    \]
    Show that this problem has a unique solution given by
    \[
    x^* = A^\top \left( AA^\top \right)^{-1} b + \left( I_n - A^\top \left( AA^\top \right)^{-1} A \right) x_0.
    \]
\end{bb}
\textbf{Solution:}\\
To solve this, we will rewrite the problem as a system of linear equations minimize $||x||$ subject to $Ax=b$. We can then apply Theorem 12.2
Let \[ y = x-x_0 \]
\[Ax = A(y+x_0) = b \implies Ay = b + Ax_0\]
Then the problem is now to find the unique vector $y^*$ that minimizes:  $||Ay = b - Ax_0||$
\\By Theorem 12.1, the unique solution $x^*$ to $Ax = b$ that minimizes the norm $||x||$ is given by 
$$x^*=A^T(AA^T)^{-1}b$$
Applying this theorem to the system $Ay = b - Ax_0$, we find that
\begin{align*}
    y^* &= A^T(AA^T)^{-1}(b+Ax_0) \\
    x^* &= y^* + x_0 =  A^T(AA^T)^{-1}(b+Ax_0) + x_0 \\
    &= A^T(AA^T)^{-1}b + A^T(AA^T)^{-1}Ax_0 + x_0 \\
    &= A^T(AA^T)^{-1}b + (I_n - A^T(AA^T)^{-1}A)x_0
\end{align*}

\bigbreak

\begin{bb}
\item Prove the following properties of generalized inverses:
\begin{enumerate}
    \item[(a)] $(A^\top)^\dagger = (A^\dagger)^\top$.
    \item[(b)] $(A^\dagger)^\dagger = A$.
\end{enumerate}    
\end{bb}

\textbf{Solution:}\\
\begin{enumerate}
    \item[(a)] 
Recall that the pseudoinverse, $A^\dagger \in \mathbb{R}^{mxn}$ of $A$ can be defined as $$A^\dagger = (A^TA)^{-1}A^T = (AA^T)^{-1}A^T$$
So, using this definition, we have:
\begin{align*}
    (A^T)^\dagger &= (AA^T)^{-1}A^T \\
    (A^\dagger)^T &= (AA^T)^{-1}A^T \\ 
    (A^T)^\dagger &=  (A^\dagger)^T
\end{align*}    
    \item[(b)] 
\end{enumerate}

\end{enumerate}

\end{document}
